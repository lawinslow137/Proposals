\section{Postdoctoral Researcher Mentoring Plan}
The postdoctoral researcher that will be supported by this award is a key member of the group. As they will spend a significant amount of time onsite commissioning the slow control system, they will provide an information conduit from site to UCLA. They will also be responsible for coordinating the analysis of exotic double-beta decay modes. The combination of well-defined hardware and analysis goals in addition to data from the full CUORE within the time period of their appointment will provide them with results to launch them nicely into a future academic or industrial position.

More specifically, upon their arrival an in-depth conversation between them and the PI will take place. This conversation will outline their career goals and the expectations of the PI. This should also include a discussion of issues related to their working style, interaction with coworkers, lab safety, and expectations of publications and other documentation to allow the continuation of their work. A similar conversation will take place every 3-4 months to ensure that the postdoc is making good progress towards both their and the PI's goals. As the postdoc advances, this conversation will focus more and more on their career goals.

The main tasks of this postdoc are the slow control system and the analysis. On the PI's side, there are three main skills that as a mentor the PI must  ensure that the postdoc gains during their appointment: the ability to teach and mentor students, the ability to present their research in presentations and publications and the ability to prepare a grant. With this in mind, the success of this mentoring plan with be assessed by meeting the following goals during the period of the award. The postdoc will be the advisor for at least one completed undergraduate thesis. Their work will result in at least one publication on exotic double-beta decay searches and at least one hardware oriented publication on the slow monitoring system. This work will also be presented at conferences. Finally, the postdoc will help draft the next iteration of this proposal.  The assessment of the career mentoring is more difficult due to the timing of this award and fluctuations in the job market; however, a reasonable goal in this three year period is that the postdoc is either applying for their desired next appointment or within a year of doing so.
 