\section{Postdoctoral Researcher Mentoring Plan}
The postdoctoral researcher supported by this award is Dr. Kevin Hickerson. He did his graduate work at Caltech on the ultra-cold neutron (UCN) experiments at Los Alamos National Laboratory (LANL). After a short postdoc continuing this work, he started in the Winslow group in August 2013. He is a key member of the group since he will spend a significant amount of time onsite commissioning the slow monitoring system. He will also lead the MC development in the U.S. and will coordinate the analysis of exotic double-beta decay modes. These well-defined leadership position in hardware and software in addition to the first data from the full CUORE detector within the time period of his appointment, should nicely launch him into a future academic or industrial position.

During his interview and again upon his arrival at UCLA, there were several in-depth conversations about his career goals and the expectations of the PI especially related to the schedule of the tasks outlined in this proposal.  This conversation also covered other details of his working style including his need to work remotely one day a week and to plan travel well in advance for family reasons. A similar conversation will take place every 3-4 months to ensure that Hickerson is making good progress towards both his and the PI's goals. As he advances, this conversation will focus more and more on their career goals.

Hickerson would like to become faculty at a research I university. His background in UCN physics is unique and complementary to double-beta decay physics. Hickerson has been very involved in a proposal to measure the neutron lifetime at LANL, UCN$\tau$.  The two most recent measurements of the neutron lifetime significantly disagree with each other. This discrepancy is large enough to challenge both the standard model of particle physics and big bang nucleosynthesis. Thus, there is significant motivation to measure the neutron lifetime to greater accuracy than and with differing systematic effects from previous experiments. A program that combines the next generation of CUORE with UCN$\tau$, would make Hickerson an attractive faculty candidate and would distinguish him from both the PI and the other CUORE postdocs. For these reason, the PI is encouraging him to maintain his involvement in the proposal at a low level. 

The vast majority of his work will be the slow control system, MC and the analysis. On the PI's side, there are three main skills that as a mentor the PI must  ensure that the postdoc gains during their appointment: the ability to teach and mentor students, the ability to present their research in presentations and publications and the ability to prepare a grant proposal. With this in mind, the success of this mentoring plan with be assessed by meeting the following goals during the period of the award. Hickerson will be the advisor for at least one completed undergraduate thesis. His work will result in at least one publication on exotic double-beta decay searches and at least one hardware oriented publication on the slow monitoring system. This work will also be presented at conferences. Finally, Hickerson has helped draft the analysis section of this proposal and he will be involved in the crafting of the next iteration of this proposal.  The assessment of the career mentoring is more difficult due to the timing of this award and fluctuations in the job market; however, a reasonable goal in this three year period is that he will be either applying for their desired next appointment or within a year of doing so.
 
 %Two of the most recent measurements of the neutron lifetime signiffigantly dissagree with each other. This discrepancy is large enough to challenge both the standard model of particle physics and big bang nucleosynthesis. Thus, there is significant motivation to measure the neutron lifetime to greater accuracy than and with differing systematic effects from previous experiments. The UCN$\tau$ experiment at Los Alamos National Laboratory, is investigating the design of such an experiment using ultracold neutrons suspended in a gravito-magnetic bottle. Because UCN are held in the trap without interacting with any material walls, it is believed that an experiment based on this design may reduce the error down to 0.1 seconds and help clarify the uncertainty. If a descrpency persists far enough from the standard model predicted value, this improved measurement may lead to the discovery of new physics.