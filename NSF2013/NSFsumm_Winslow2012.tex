
%%%%%%%%% SUMMARY -- 1 page, third person
% e.g:  "The PI will prove" not "I will prove"

%\required{Project Summary}
%This award will support a programs of research into the basic properties of the neutrino especially the possible Majorana nature of this elusive particle. The only feasible experiments to address this issue are searches for neutrinoless double beta decay. If this rare nuclear process is observed then the neutrino is a Majorana particle and the doors are open for theories that use neutrinos to generate the matter antimatter asymmetry in the universe.   

%\required{Intellectual Merit}
% This is why your project is interesting and will help further
% knowledge in the field of mathematics. 
%This award will be used to support students and a postdoc at UCLA on the CUORE (Cryogenic Underground Observatory for Rare Events) experiment. This experiment searches for the neutrinoless double beta decay of \isomain~using TeO$_{2}$ crystals operated as bolometers. This is a potentially transformative detector technology that may move beyond double beta decay searches to beta decay endpoint measurements, dark matter searches and beyond; therefore the general intellectual merit of this proposal is the growth of US expertise in this technology. More specifically, the group sponsored by this award will develop the slow monitoring system for the experiment and in addition to the main analysis will specialize in the search for more exotic neutrinoless double beta modes like that those through Majoran emission.

%The main focus of this proposal is the search for neutrinoless double beta decay in \isomain with the CUORE (Cryogenic Underground Observatory for Rare Events) experiment. The experiment uses TeO$_{2}$ crystals operated as bolometers, a potentially transformative detector technology that may move beyond double beta decay searches to beta decay endpoint measurements, dark matter searches and beyond. The award will also sponsor the development of future neutrino experiments including experiments that could address the Majorana nature of the neutrino in the normal hierarchy, the existence of sterile neutrinos, and CP violation in neutrino oscillations.

%\required{Broader Impacts}
% There are 4 kinds of broader impacts.
% 1. advance discovery and understanding while promoting teaching,
% training and learning
% 2. broaden the participation of underrepresented groups
% 3. disseminated broadly to enhance scientific and technological
% understanding
% 4. benefits of the proposed activity to society
%The broader impact of this proposal is the training of a next generation of graduate students in low energy nuclear physics. In addition, the PI has a track record of working to increase the participation of women and minorities in physics. This work will become part of the greater work of the group through undergraduate and graduate research opportunities, mentoring of other students within the department and public outreach.

%
%Janet's suggested rewrite.
%
This award will support a program of research into the basic properties of the neutrino, with special focus on the possible ``Majorana nature'' of this elusive particle. In the near future, the only feasible experiments to address this issue are searches for neutrinoless double-beta decay. If this as-yet-unobserved nuclear process is discovered,  then the neutrino is, by necessity, a Majorana particle.  If this is the case, the doors are open for theories that use neutrinos to generate the matter-antimatter asymmetry in the universe.
 
The award will be used to support a PI, graduate and undergraduate students and a postdoc at the University of California, Los Angeles (UCLA), on the CUORE (Cryogenic Underground Observatory for Rare Events) experiment. The experiment searches for the neutrinoless double-beta decay of $^{130}$Te using TeO$_2$ crystals operated as bolometers. This is a potentially transformative detector technology, which has the best reach of any search within the next five years and promise to move beyond double-beta decay searches to beta decay endpoint measurements, and potentially direct searches for dark matter.
 
The PI brings important expertise to CUORE in slow monitoring, in the simulation of backgrounds, and in the analysis of low-energy neutrino experiments. With this award, the group will develop the slow monitoring system for the experiment and in addition to the main analysis will specialize in the search for more exotic neutrinoless double-beta modes such as those through Majoran emission.
 
Thus, the {\bf intellectual merit} of this proposal is threefold.  First, the scientific importance of neutrinoless double-beta decay is very high.  Second, this promotes the growth of expertise in this novel type of bolometer technology in the US.  Lastly, it will bring important expertise in slow monitoring to the CUORE experiment.
 
The PI has a strong track record of working to increase the participation of women and minorities in physics. This work will become part of the greater endeavors of the group through undergraduate and graduate research opportunities, mentoring of other students within the department and public outreach. The training of a next generation of students in low-energy nuclear physics represents the {\bf broader impact} of this proposal.
 