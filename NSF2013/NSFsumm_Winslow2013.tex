\noindent
{\bf Intellectual Merit}
\\
Neutrino physics has produced many exciting results in the last decade, yet many basic  questions remain about the nature of these particles. The most fundamental is whether the neutrino is its own antiparticle, a Majorana particle. A Majorana neutrino would have profound consequences for particle physics and cosmology, including a possible explanation for the matter-antimatter asymmetry in the universe. The only feasible experiments to address this issue are searches for neutrinoless double-beta decay ($0\nu\beta\beta$).  The most likely mechanism for this process is light Majorana neutrino exchange although more exotic decay modes including the emission of a boson, the Majoron, are also possible.
 
This award will be used to support a new group at the University of California, Los Angeles (UCLA), consisting of the PI, graduate and undergraduate students and a postdoc on the CUORE (Cryogenic Underground Observatory for Rare Events) experiment. The experiment's main measurement is the search for the neutrinoless double-beta decay of $^{130}$Te using TeO$_2$ crystals operated as bolometers.  The experiment has great discovery potential within the next five years if the neutrino follows the inverted mass hierarchy. This is also a potentially transformative detector technology which shows much promise to move beyond double-beta decay searches to beta decay endpoint measurements, and potentially direct searches for dark matter.
 
The PI brings important expertise to CUORE in slow monitoring, in the simulation of backgrounds, and in the analysis of low-energy neutrino experiments. With this award, the group will develop the slow monitoring system for the experiment and in addition to the main analysis will specialize in the search for more exotic neutrinoless double-beta modes such as those through Majoron emission. The additional manpower comes at a critical time as the experiment transitions from construction to commissioning and data analysis.
\\ 

\noindent
{\bf Broader Impacts}
\\ 
 The greatest impact of this proposal is the training of a next generation of students in low-energy nuclear physics. In addition, the PI has a strong track record of working to increase the participation of women and minorities in physics. This work will become part of the greater endeavors of the group through undergraduate and graduate research opportunities, mentoring of other students within the department and public outreach.  The simulation work naturally leads to the development of a web-based event viewer that will allow the public to explore the CUORE detector and view events.
 