\section{Data Management Plan}
The data collected as part of this award will be collected by three different collaborations: CUORE, KamLAND, and Double Chooz. Each of these experiments use waveform digitizers to digitize the signals from the detectors. This means that the raw data is relatively large and even to members of these collaborations not very useful until at least basic analysis and calibration is done. The goal of the data management plan is not to provide the public with every single gigabyte.  The goal is to provide the public the data that is needed to reproduce the result and make sure this data is properly curated for the coming decades.

All of these experiments are analyzing energy spectra, therefore what is needed to reproduce the result are the final candidate events either in a tree or a histogram, the background estimates, and the systematic uncertainties. These data and the documentation needed to interpret them will be provided on a website linked to the main CUORE website. This will make the data easy to locate and make it easy for people to contact the experts for more information if needed.

The main issue is making sure that the data and documentation is available long after the experiment is over. It is also important to be able to properly cite this data when others use it in their research. The UC3: University of California Curation Center provides resources to do this through Merritt, a repository for long term data management. The data and documentation are uploaded into the system and receive a permanent web area that can be linked to the CUORE website and cited in papers. This system is similar to other systems like DSpace at MIT that are designed to curate the increasing number of large digital data sets all disciplines are now producing.

The other outcome of this award will be software for controlling the slow monitor system. This will be based on an open source code base like EPICS. The new software especially drivers for specific devices needs to be included in the parent code base. The group will work with the managers of the chosen software package to make sure it is included in future releases. The code and its documentation can also be stored in Merritt. This ensures that it will have a persistent citable link for publication.

Internal to these collaboration, software is saved and tracked in an svn or CVS systems. The raw data is stored on computing clusters devoted to these experiments. A backup copy of the data is stored in other locations like Lawrence Berkeley National Lab's PDSF. The UCLA group has access to these facilities in addition to nodes and disk space on their own Hoffman2 cluster.