\section{Data Management Plan}
The CUORE experiment uses waveform digitizers to digitize the signals coming from the crystals. This means that the raw data is relatively large and even to members of the collaboration not very useful until at least basic analysis and calibration is done. The goal of the data management plan is not to provide the public with every single gigabyte.  The goal is to provide the public the data that is needed to reproduce the result and make sure this data is properly curated for the coming decades.

In order to reproduce the CUORE result,  one needs the candidate events either in a tree or a histogram, the background estimates, and the systematic uncertainties. These data and the documentation needed to interpret them will be provided on a website linked to the main CUORE website. This will make the data easy to locate and make it easy for people to contact the experts for more information if needed.

The main issue is making sure that the data and documentation is available long after the experiment is over. It is also important to be able to properly cite this data when others use it in their research. The UC3: University of California Curation Center provides resources to do this through Merritt, a repository for long term data management. The data and documentation are uploaded into the system and receive a permanent web area that can be linked to the CUORE website and cited in papers. This system is similar to other systems like DSpace at MIT that are designed to curate the increasing number of large digital data sets all disciplines are now producing.

The other outcome of this award will be software for controlling the slow monitor system. This will be based on an open source code base TANGO. The new software especially drivers for specific devices needs to be included in the parent code base. The group will work with the TANGO developers to make sure the code is included in future releases. The code and its documentation can also be stored in Merritt. This ensures that it will have a persistent citable link for publication.

Internal to CUORE, software is saved and tracked using svn. The Winslow group also uses github to manage their software including the tex of proposals like this one. The raw data from CUORE is stored on the Rome computing cluster. A backup copy of the data is stored at Lawrence Berkeley National Lab's PDSF. The UCLA group has access to these facilities in addition to nodes and disk space on their own Hoffman2 cluster.