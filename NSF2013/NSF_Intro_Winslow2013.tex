
%%%%%%%%% PROPOSAL -- 15 pages (including Prior NSF Support)

\section{Introduction}
In the last decade, our knowledge of the neutrino has grown exponentially. Low energy neutrino $\sim$MeV experiments such as KamLAND have been instrumental in proving that neutrinos oscillate and therefore have mass. Although our picture of the physics is much clearer now, many questions remain in neutrino physics. One of the most tantalizing is the question of the Majorana nature of the neutrino or stated differently is the neutrino its own antiparticle. If the neutrino is Majorana then in models of Leptogenesis, the neutrino is responsible for the matter-antimatter  asymmetry we observe in the universe today. This is why both the APS Multidivisional Neutrino Study from 2004 \cite{numatrix}  and this year's National Academy report on nuclear physics\cite{national2012Nuclear}, highlight this as a key question to be addressed in the next generation of neutrino experiments.

The only feasible experimental probes of the Majorana nature of the neutrino are low energy low energy neutrino experiments searching for the rare nuclear process neutrinoless double beta decay (\zeronu). The next generation of experiments has started to take data with first results from EXO\cite{EXO2012} and KamLAND-Zen\cite{KZ2nu} presented in the last year. These experiments are designed to either confirm or refute the claimed observation in \isoge\cite{KKK2006}. The first experiment designed to have sensitivity to push beyond these bounds, and for the first time push into the expected region for neutrino masses following the inverted-hierarchy, is the CUORE experiment. 

The CUORE experiment is a two-staged experiment building on the experience of the successful CUORICINO experiment\cite{CC2008}. It uses tellurium crystals instrumented as bolometers to search for \zeronu~in \isomain. The first stage CUORE-0 started in the Fall of 2012 with the commencement of data taking from the first production tower operating in the original CUORICINO cryostat. This makes it an excellent time for a young PI to join the experiment, gain experience in this unique detector technology, and make a valuable contribution to both the hardware and the analysis.

The PI of this proposal has considerable experience in low energy neutrino physics from their work on the KamLAND and Double Chooz experiments. In particular, the PI is very familiar with the modeling of radioactive backgrounds from natural radioactivity in addition to extensive experience simulating radioactive backgrounds produced in muon spallation. The final sensitivity of the \zeronu~analysis depends almost entirely on the understanding of the backgrounds, therefore the PI's understanding of backgrounds transfers very effectively of the CUORE analysis.

The collaboration has identified the unification of the various slow control systems as the piece of the project that needs more resources. Because CUORE represents a leap forward in size for this technology, the tools developed for CUORICINO are insufficient, and the system needs to be engineered from the bottom up, while making use of new software packages designed for operating detectors of this scale. Special attention needs to be taken with the interface of the slow control data to the bolometer data and the detector Monte Carlo, since changes in the environment can effect the background reconstruction. The PI performed the same job on Double Chooz, so brings with them experience from commissioning and operating that detector, and is well positioned to lead this effort.    

The hardware and analysis tasks outlined in this proposal fit in nicely with the PI's skills, and the project's needs. This also fits in nicely with the other work being done at UCLA  by Professor Huan Huang's group on the front-end electronics. The combination of fronted electronics with the slow control will help unify the operation of the detector and provides a quorum  for analysis at UCLA. Finally, the timescale of the proposed work fits well with the schedule of the experiment. In year one of the grant, the slow control interfaces would be written and the data from CUORE-0 analyzed. This is followed by year two of the grant when the full CUORE detector is commissioned, and first results from CUORE coming in year three of the grant. This schedule, the details of the physics, and the proposed work are outlined further in the body of this proposal.
