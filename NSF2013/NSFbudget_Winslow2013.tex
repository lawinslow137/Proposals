\section{Budget Justification}

\subsection{Personnell}
Funds are requested to support 2 months summer salary for the Principal Investigator (Professor Lindley Winslow) for 3 years of the grant.  Funds are requested to support 1 Post Doctoral Scholar at 100\% time for 12 months for 3 years.  Funds are also requested to support 1 Graduate Student Researcher for 3 years. The GSR will work 9 academic months at 50\% time and 3 summer months at 100\% time.  Graduate student salaries are based on current UC research assistant salary scale. In addition, we are also requesting funds for 2 undergraduate student lab assistants for 3 years.

Salary costs include projected cost of living and range adjustment increases for 3 years of the grant.  Rate increase projected at 3\% per year. Benefits are added for each employee at 12.7\% for faculty academic, 19\% for Post Doctoral Scholar.  GSR benefit rate is 1.3\% academic months, 3\% summer months.  Student lab assistant
rate is 3\%.


\subsection{Materials and Supplies}
Funds are requested to cover research materials and supplies, computer hardware and software updates. This budget is \$6,000 per year. In addition \$11,000 is requested for constructing an environmental monitoring system for the CUORE experiment, please see main proposal text for budget.


\subsection{Travel}
Travel funds are requested for all members of the group. Foreign airfare is estimated at \$1,500 per trip. Domestic airfare is estimated at \$600. An average of \$1,000 per week is estimated for food and lodging. With these numbers in mind, the PI plans to make 3 foreign trips for CUORE to attend collaboration meetings and take shift. In addition the PI plans to make 2 foreign trips per year to either complete their Double Chooz work or attend KamLAND collaboration meetings and take shift. With two domestic trips to attend U.S. CUORE meetings and a conference, the PI's yearly budget is \$15,000.  The postdoc's budget is estimated similarly for 3 foreign trips and two domestic trips for a yearly budget of \$12,000.  These numbers are based on the PI's travel budget from the preceding year working on the Double Chooz experiment located in France.

In addition to the PI's and postdoc's travel, money is requested for the graduate student and undergraduate students to travel. The graduate student will spend each summer at Grand Sasso and will be expected to attend a domestic conference. The budget for this extended foreign trip and one domestic trip is \$6,000.  Similarly, \$3000 is requested for each undergraduate to make an extended foreign trip to the experimental site and attend a domestic conference. It is assumed that the undergraduates will take part in a program like Conference Experience for Undergraduates (CEU) at the annual Division of Nuclear Physics or American Physical Society meetings which will reduce the expense to the grant and will ensure a more enriching conference experience for them.


\subsection{Other Direct Costs}

The Technology Infrastructure Fee (TIF) is a consistently-applied direct charge that is assessed to each and every campus activity unit, regardless of funding source, including units identified as individual grant and contract awards. The TIF pays for campus communication services on the basis of a monthly accounting of actual usage data. These costs are charged as direct costs and are not recovered as indirect costs. The TIF rate is calculated at \$38.41 per month per FTE.

Graduate Student fee remission (GSR) in the amount of \$14,435 per student is added to each academic year. The fee remission listed in the budget reflects a 5\% annual increase per year.

\subsection{Indirect Costs}

Indirect costs have been applied to all direct costs except GSR fees, Fabrication and Equipment as per agreement with the Department of Health and Human Services dated April 27, 2011.  	      Rate applied is 26\%.
