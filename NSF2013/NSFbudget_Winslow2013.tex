\section{Budget Justification}

\subsection{Personnel}

Funds are requested to support 2 month summer salary for the Principal Investigator (Professor Lindley Winslow) for 3 years of the grant.  Funds are requested as well to support 1 Post Doctoral Scholar at 100\% time for 12 months for 3 years.  Funds are also requested to support 1 Graduate Student Researcher for 3 years. The GSR will work 9 academic months at 50\% time and 3 summer months at 100\% time. Graduate student salaries are based on current UC research assistant salary scale. In addition, we are also requesting funds for 2 undergraduate student lab assistants for 3 years. Salary costs include projected cost of living and range adjustment increases for 3 years of the grant.  Rate increase projected at 3\% per year.

Benefits are added for each employee at 12.7\% for faculty academic, 19\% for Post Doctoral Scholar.  GSR benefit rate is 1.3\% academic months, 3\% summer months.  Student lab assistant rate is 1.3\%. Fringe benefits are calculated at composite rates currently in effect for the University: \url{http://www.research.ucla.edu/ocga/sr2/ben_rate.htm}.

\subsection{Materials and Supplies}
Funds are requested to cover research materials and supplies, computer hardware and software updates. This budget is \$3,000 per year. In addition \$18,000 is requested for constructing an environmental monitoring system for the CUORE experiment, please see main proposal text for budget.


\subsection{Travel}
Travel funds are requested for all members of the group. Foreign airfare is estimated at \$1,500 per trip. Domestic airfare is estimated at \$500 except for the airfare to UC Berkeley which is estimated at \$250. An average of \$1,000 per week is estimated for food and lodging. For extended trips onsite, the cost is reduced to \$500 per week. With these numbers in mind, the domestic travel budget is \$11,000 corresponding to all group members (the PI, postdoc, graduate student and two undergraduate students) attending a domestic conference, the PI and postdoc attending a domestic collaboration meeting, and two day trips to UC Berkeley. The foreign travel budget is \$28,000 corresponding to a multi-week stay onsite for the graduate student and two multi-week stays for the postdoc, an international conference for the PI and postdoc and an international collaboration meeting for the PI and postdoc. This budget also includes one additional international collaboration meeting in Japan for the PI to nurture future efforts there. This travel schedule and the associated costs are based on the PI's travel in the last year and experience working on other international collaborations.


\subsection{Other Direct Costs}
The Technology Infrastructure Fee (TIF) is a consistently-applied direct charge that is assessed to each and every campus activity unit, regardless of funding source, including units identified as individual grant and contract awards. The TIF pays for campus communication services on the basis of a monthly accounting of actual usage data. These costs are charged as direct costs and are not recovered as indirect costs. The TIF rate is calculated at \$35.42 per month per FTE:\url{https://www.it.ucla.edu/support/billing-questions/2013-14-billing-rates-and-structure-announcement}.

Graduate Student fee remission (GSR) in the amount of \$14,830 per student is added to each academic year. The fee remission listed in the budget reflects a 5\% annual increase per year and  exempt from F\&A cost: \url{http://www.gdnet.ucla.edu/gss/library/1314remissionsgsr.pdf}.


\subsection{Indirect Costs}
Indirect costs have been applied to all direct costs except GSR fees, Fabrication and Equipment as per agreement with the Department of Health and Human Services dated April 27, 2011. Rate applied is 26\%: \url{http://www.research.ucla.edu/ocga/Documents/F_A_Rate_Agreement_4-27-11.pdf}.

